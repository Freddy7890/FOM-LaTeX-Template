%-----------------------------------
% Define document and include general packages
%-----------------------------------
% Tabellen- und Abbildungsverzeichnis stehen normalerweise nicht im
% Inhaltsverzeichnis. Gleiches gilt für das Abkürzungsverzeichnis (siehe unten).
% Manche Dozenten bemängeln das. Die Optionen 'listof=totoc,bibliography=totoc'
% geben das Tabellen- und Abbildungsverzeichnis im Inhaltsverzeichnis (toc=Table
% of Content) aus.
% Da es aber verschiedene Regelungen je nach Dozent geben kann, werden hier
% beide Varianten dargestellt.
\documentclass[12pt,oneside,titlepage,listof=totoc,bibliography=totoc]{scrartcl}
%\documentclass[12pt,oneside,titlepage]{scrartcl}

%-----------------------------------
% Dokumentensprache
%-----------------------------------
%\def\FOMEN{}% Auskommentieren um die Dokumentensprache auf englisch zu ändern
\newif\ifde
\newif\ifen

%-----------------------------------
% Meta informationen
%-----------------------------------
%-----------------------------------
% Meta Informationen zur Arbeit
%-----------------------------------

% Autor
\newcommand{\myAutor}{Max Mustermann}

% Adresse
\newcommand{\myAdresse}{Heidestra\ss e 17 \\ \> \> \> 51147 Köln}

% Titel der Arbeit
\newcommand{\myTitel}{LATEX-Vorlage - mit Biblatex}

% Betreuer
\newcommand{\myBetreuer}{Prof. Dr. Peter Lustig}

% Lehrveranstaltung
\newcommand{\myLehrveranstaltung}{Modul Nr. 1}

% Matrikelnummer
\newcommand{\myMatrikelNr}{123456}

% Ort
\newcommand{\myOrt}{Düsseldorf}

% Datum der Abgabe
\newcommand{\myAbgabeDatum}{\today}

% Semesterzahl
\newcommand{\mySemesterZahl}{7}

% Name der Hochschule
\newcommand{\myHochschulName}{FOM Hochschule für Oekonomie \& Management}

% Standort der Hochschule
\newcommand{\myHochschulStandort}{Düsseldorf}

% Studiengang
\newcommand{\myStudiengang}{Wirtschaftsinformatik}

% Art der Arbeit
\newcommand{\myThesisArt}{Bachelor Thesis}

% Zu erlangender akademische Grad
\newcommand{\myAkademischerGrad}{Bachelor of Science (B.Sc.)}

% Firma
\newcommand{\myFirma}{Mustermann GmbH}

% Definition der Sprache
\ifdefined\FOMEN
%Englisch
\entrue
\usepackage[english]{babel}
\else
%Deutsch
\detrue
\usepackage[ngerman]{babel}
\fi


\newcommand{\langde}[1]{%
   \ifde\selectlanguage{ngerman}#1\fi}
\newcommand{\langen}[1]{%
   \ifen\selectlanguage{english}#1\fi}
\usepackage[utf8]{luainputenc}
\langde{\usepackage[babel,german=quotes]{csquotes}}
\langen{\usepackage[babel,english=british]{csquotes}}
\usepackage[T1]{fontenc}
\usepackage{fancyhdr}
\usepackage{fancybox}
\usepackage[a4paper, left=4cm, right=2cm, top=4cm, bottom=2cm]{geometry}
\usepackage{graphicx}
\usepackage{colortbl}
\usepackage[capposition=top]{floatrow}
\usepackage{array}
\usepackage{float}      %Positionierung von Abb. und Tabellen mit [H] erzwingen
\usepackage{footnote}
% Darstellung der Beschriftung von Tabellen und Abbildungen (Leitfaden S. 44)
% singlelinecheck=false: macht die Caption linksbündig (statt zentriert)
% labelfont auf fett: (Tabelle x.y:, Abbildung: x.y)
% font auf fett: eigentliche Bezeichnung der Abbildung oder Tabelle
% Fettschrift laut Leitfaden 2018 S. 45
\usepackage[singlelinecheck=false, labelfont=bf, font=bf]{caption}
\usepackage{caption}
\usepackage{enumitem}
\usepackage{amssymb}
\usepackage{mathptmx}
%\usepackage{minted} %Kann für schöneres Syntax Highlighting genutzt werden. ACHTUNG: Python muss installiert sein.
\usepackage[scaled=0.9]{helvet} % Behebt, zusammen mit Package courier, pixelige Überschriften. Ist, zusammen mit mathptx, dem times-Package vorzuziehen. Details: https://latex-kurs.de/fragen/schriftarten/Times_New_Roman.html
\usepackage{courier}
\usepackage{amsmath}
\usepackage[table]{xcolor}
\usepackage{marvosym}			% Verwendung von Symbolen, z.B. perfektes Eurozeichen

\renewcommand\familydefault{\sfdefault}
\usepackage{ragged2e}

% Mehrere Fussnoten nacheinander mit Komma separiert
\usepackage[hang,multiple]{footmisc}
\setlength{\footnotemargin}{1em}

% todo Aufgaben als Kommentare verfassen für verschiedene Editoren
\usepackage{todonotes}

% Verhindert, dass nur eine Zeile auf der nächsten Seite steht
\setlength{\marginparwidth}{2cm}
\usepackage[all]{nowidow}

%-----------------------------------
% Farbdefinitionen
%-----------------------------------
\definecolor{darkblack}{rgb}{0,0,0}
\definecolor{dunkelgrau}{rgb}{0.8,0.8,0.8}
\definecolor{hellgrau}{rgb}{0.0,0.7,0.99}
\definecolor{mauve}{rgb}{0.58,0,0.82}
\definecolor{dkgreen}{rgb}{0,0.6,0}

%-----------------------------------
% Pakete für Tabellen
%-----------------------------------
\usepackage{epstopdf}
\usepackage{nicefrac} % Brüche
\usepackage{multirow}
\usepackage{rotating} % vertikal schreiben
\usepackage{mdwlist}
\usepackage{tabularx}% für Breitenangabe

%-----------------------------------
% sauber formatierter Quelltext
%-----------------------------------
\usepackage{listings}
% JavaScript als Sprache definieren:
\lstdefinelanguage{JavaScript}{
	keywords={break, super, case, extends, switch, catch, finally, for, const, function, try, continue, if, typeof, debugger, var, default, in, void, delete, instanceof, while, do, new, with, else, return, yield, enum, let, await},
	keywordstyle=\color{blue}\bfseries,
	ndkeywords={class, export, boolean, throw, implements, import, this, interface, package, private, protected, public, static},
	ndkeywordstyle=\color{darkgray}\bfseries,
	identifierstyle=\color{black},
	sensitive=false,
	comment=[l]{//},
	morecomment=[s]{/*}{*/},
	commentstyle=\color{purple}\ttfamily,
	stringstyle=\color{red}\ttfamily,
	morestring=[b]',
	morestring=[b]"
}

\lstset{
	%language=JavaScript,
	numbers=left,
	numberstyle=\tiny,
	numbersep=5pt,
	breaklines=true,
	showstringspaces=false,
	frame=l ,
	xleftmargin=5pt,
	xrightmargin=5pt,
	basicstyle=\ttfamily\scriptsize,
	stepnumber=1,
	keywordstyle=\color{blue},          % keyword style
  	commentstyle=\color{dkgreen},       % comment style
  	stringstyle=\color{mauve}         % string literal style
}

%-----------------------------------
%Literaturverzeichnis Einstellungen
%-----------------------------------

% Biblatex

\usepackage{url}
\urlstyle{same}

%%%% Neuer Leitfaden (2018)
\usepackage[
backend=biber,
style=ext-authoryear-ibid, % Auskommentieren und nächste Zeile einkommentieren, falls "Ebd." (ebenda) nicht für sich-wiederholende Fussnoten genutzt werden soll.
%style=ext-authoryear,
maxcitenames=3,	% mindestens 3 Namen ausgeben bevor et. al. kommt
maxbibnames=999,
mergedate=false,
date=iso,
seconds=true, %werden nicht verwendet, so werden aber Warnungen unterdrückt.
urldate=iso,
innamebeforetitle,
dashed=false,
autocite=footnote,
doi=false,
useprefix=true, % 'von' im Namen beachten (beim Anzeigen)
mincrossrefs = 1
]{biblatex}%iso dateformat für YYYY-MM-DD

%weitere Anpassungen für BibLaTex
\usepackage{xpatch}

\setlength\bibhang{1cm}

%%% Weitere Optionen
%\boolitem[false]{citexref} %Wenn incollection, inbook, inproceedings genutzt wird nicht den zugehörigen parent auch in Literaturverzeichnis aufnehmen

%Aufräumen die Felder werden laut Leitfaden nicht benötigt.
\AtEveryBibitem{%
\ifentrytype{book}{
    \clearfield{issn}%
    \clearfield{doi}%
    \clearfield{isbn}%
    \clearfield{url}
    \clearfield{eprint}
}{}
\ifentrytype{collection}{
  \clearfield{issn}%
  \clearfield{doi}%
  \clearfield{isbn}%
  \clearfield{url}
  \clearfield{eprint}
}{}
\ifentrytype{incollection}{
  \clearfield{issn}%
  \clearfield{doi}%
  \clearfield{isbn}%
  \clearfield{url}
  \clearfield{eprint}
}{}
\ifentrytype{article}{
  \clearfield{issn}%
  \clearfield{doi}%
  \clearfield{isbn}%
  \clearfield{url}
  \clearfield{eprint}
}{}
\ifentrytype{inproceedings}{
  \clearfield{issn}%
  \clearfield{doi}%
  \clearfield{isbn}%
  \clearfield{url}
  \clearfield{eprint}
}{}
}

\renewcommand*{\finentrypunct}{}%Kein Punkt am ende des Literaturverzeichnisses

\renewcommand*{\newunitpunct}{\addcomma\space}
\DeclareDelimFormat[bib,biblist]{nametitledelim}{\addcolon\space}
\DeclareDelimFormat{titleyeardelim}{\newunitpunct}
%Namen kursiv schreiben
\renewcommand*{\mkbibnamefamily}{\mkbibemph}
\renewcommand*{\mkbibnamegiven}{\mkbibemph}
\renewcommand*{\mkbibnamesuffix}{\mkbibemph}
\renewcommand*{\mkbibnameprefix}{\mkbibemph}

% Die Trennung mehrerer Autorennamen erfolgt durch Kommata.
% siehe Beispiele im Leitfaden S. 16
% Die folgende Zeile würde mit Semikolon trennen
%\DeclareDelimFormat{multinamedelim}{\addsemicolon\addspace}

%Delimiter für mehrere und letzten Namen gleich setzen
\DeclareDelimAlias{finalnamedelim}{multinamedelim}

\DeclareNameAlias{default}{family-given}
\DeclareNameAlias{sortname}{default}  %Nach Namen sortieren


\DeclareFieldFormat{editortype}{\mkbibparens{#1}}
\DeclareDelimFormat{editortypedelim}{\addspace}
\DeclareFieldFormat{translatortype}{\mkbibparens{#1}}
\DeclareDelimFormat{translatortypedelim}{\addspace}
\DeclareDelimFormat[bib,biblist]{innametitledelim}{\addcomma\space}

\DeclareFieldFormat*{citetitle}{#1}
\DeclareFieldFormat*{title}{#1}
\DeclareFieldFormat*{booktitle}{#1}
\DeclareFieldFormat*{journaltitle}{#1}

\xpatchbibdriver{online}
  {\usebibmacro{organization+location+date}\newunit\newblock}
  {}
  {}{}

\DeclareFieldFormat[online]{date}{\mkbibparens{#1}}
\DeclareFieldFormat{urltime}{\addspace #1\addspace \langde{Uhr}\langen{MEZ}}
\DeclareFieldFormat{urldate}{%urltime zu urldate hinzufügen
  [\langde{Zugriff}\langen{Access}\addcolon\addspace
  #1\printfield{urltime}]
}
\DeclareFieldFormat[online]{url}{<\url{#1}>}
\renewbibmacro*{url+urldate}{%
  \usebibmacro{url}%
  \ifentrytype{online}
    {\setunit*{\addspace}%
     \iffieldundef{year}
       {\printtext[date]{keine Datumsangabe }}
       {\usebibmacro{date}}%
       \setunit*{\addspace}%
       \usebibmacro{urldate}}%
    {}%
  }

%Verhindern, dass bei mehreren Quellen des gleichen Autors im gleichen Jahr
%Buchstaben nach der Jahreszahl angezeigt werden wenn sich das Keyword in usera unterscheidet.
\DeclareExtradate{
  \scope{
    \field{labelyear}
    \field{year}
    }
    \scope{
      \field{usera}
     }
}

%% Anzeige des Jahres nach dem Stichwort (usera) im Literaturverzeichnis
%% Wenn das Jahr bei Online-Quellen nicht explizit angegeben wurde, wird nach
%% dem Stichwort 'o. J.' ausgegeben. Nach der URL steht dann 'keine
%% Datumsangabe'. Ist das Jahr definiert, wird es an beiden Stellen ausgegeben.
%% Das Zugriffsdatum (urldate) spielt hier keine Rolle.
%% Für Nicht-Online-Quellen wird nichts geändert.
\renewbibmacro*{date+extradate}{%
  \printtext[parens]{%
    \printfield{usera}%
    \setunit{\printdelim{titleyeardelim}}%
    \ifentrytype{online}
       {\setunit*{\addspace\addcomma\addspace}%
         \iffieldundef{year}
           {\bibstring{nodate}}
       {\printlabeldateextra}}%
       {\printlabeldateextra}}}

%% Anzeige des Jahres nach dem Stichwort (usera) in der Fussnote
%% das Stichwort hat der Aufrufer hier schon ausgegeben.
%% siehe auch Kommentar zu: \renewbibmacro*{date+extradate}
\renewbibmacro*{cite:labeldate+extradate}{%
    \ifentrytype{online}
       {\setunit*{\addspace\addcomma\addspace}%
         \iffieldundef{year}
           {\bibstring{nodate}}
       {\printlabeldateextra}}%
       {\printlabeldateextra}}


\DefineBibliographyStrings{german}{
  nodate    = {{}o.\adddot\addspace J\adddot},
  andothers = {et\addabbrvspace al\adddot}
}
\DefineBibliographyStrings{english}{
  nodate    = {{}n.\adddot\addspace d\adddot},
  andothers = {et\addabbrvspace al\adddot}
}
\DeclareSourcemap{
  \maps[datatype=bibtex]{
    \map{
      \step[notfield=translator, final]
      \step[notfield=editor, final]
      \step[fieldset=author, fieldvalue={{{\langde{o\noexpand\adddot\addspace V\noexpand\adddot}\langen{Anon}}}}]
    }
    \map{
      \pernottype{online}
      \step[fieldset=location, fieldvalue={\langde{o\noexpand\adddot\addspace O\noexpand\adddot}\langen{s\noexpand\adddot I\noexpand\adddot}}]
    }
  }
}

\renewbibmacro*{cite}{%
  \iffieldundef{shorthand}
    {\ifthenelse{\ifnameundef{labelname}\OR\iffieldundef{labelyear}}
       {\usebibmacro{cite:label}%
        \setunit{\printdelim{nonametitledelim}}}
       {\printnames{labelname}%
        \setunit{\printdelim{nametitledelim}}}%
     \printfield{usera}%
     \setunit{\printdelim{titleyeardelim}}%
     \usebibmacro{cite:labeldate+extradate}}
    {\usebibmacro{cite:shorthand}}}

    \renewcommand*{\jourvoldelim}{\addcomma\addspace}% Trennung zwischen journalname und Volume. Sonst Space; Laut Leitfaden richtig
    %Aufgrund der Änderung bzgl des Issues 169 in der thesis_main.tex musste ich die Zeile auskommentieren. Konnte aber das Verhalten, dass die Fußnoten grün sind, im nachhinein nicht feststellen.
    %\hypersetup{hidelinks} %sonst sind Fußnoten grün. Dadurch werden Links allerdings nicht mehr farbig dargestellt

\renewbibmacro*{journal+issuetitle}{%
  \usebibmacro{journal}%
  \setunit*{\jourvoldelim}%
  \iffieldundef{series}
    {}
    {\setunit*{\jourserdelim}%
     \printfield{series}%
     \setunit{\servoldelim}}%
  \iffieldundef{volume}
    {}
    {\printfield{volume}}
  \iffieldundef{labelyear}
  {}
  {
  (\thefield{year}) %Ansonsten wird wenn kein Volume angegeben ist ein Komma vorangestellt
  }
  \setunit*{\addcomma\addspace Nr\adddot\addspace}
  \printfield{number}
  \iffieldundef{eid}
  {}
  {\printfield{eid}}
}

% Postnote ist der Text in der zweiten eckigen Klammer bei einem Zitat
% wenn es keinen solchen Eintrag gibt, dann auch nicht ausgeben, z.B. 'o. S.'
% Wenn man das will, kann man das 'o. S.' ja explizit angeben. Andernfalls steht
% sonst auch bei Webseiten 'o. S.' da, was laut Leitfaden nicht ok ist.
\renewbibmacro*{postnote}{%
  \setunit{\postnotedelim}%
  \iffieldundef{postnote}
    {} %{\printtext{\langde{o.S\adddot}\langen{no page number}}}
    {\printfield{postnote}}}

% Abstand bei Änderung Anfangsbuchstabe ca. 1.5 Zeilen
\setlength{\bibinitsep}{0.75cm}

% nur in den Zitaten/Fussnoten den Vornamen abkürzen (nicht im
% Literaturverzeichnis)

\DeclareDelimFormat{nonameyeardelim}{\addcomma\space}
\DeclareDelimFormat{nameyeardelim}{\addcomma\space}

\renewbibmacro*{cite}{%
  \iffieldundef{shorthand}
    {\ifthenelse{\ifciteibid\AND\NOT\iffirstonpage}
       {\usebibmacro{cite:ibid}}
    {\printtext[bibhyperref]{\ifthenelse{\ifnameundef{labelname}\OR\iffieldundef{labelyear}}
       {\usebibmacro{cite:label}%
        \setunit{\printdelim{nonameyeardelim}}}
      {\toggletrue{abx@bool@giveninits}%
        \printnames[family-given]{labelname}%
        \setunit{\printdelim{nameyeardelim}}}%
      \printfield{usera}%
      \setunit{\printdelim{titleyeardelim}}%
     \usebibmacro{cite:labeldate+extradate}}}}
   {\usebibmacro{cite:shorthand}}}


%%%%% Alter Leitfaden. Ggf. Einkommentieren und Bereich hierüber auskommentieren
%\usepackage[
%backend=biber,
%style=numeric,
%citestyle=authoryear,
%url=false,
%isbn=false,
%notetype=footonly,
%hyperref=false,
%sortlocale=de]{biblatex}

%weitere Anpassungen für BibLaTex
%\input{skripte/modsBiblatex}

%%%% Ende Alter Leitfaden

%Bib-Datei einbinden
\addbibresource{literatur/literatur.bib}

% Zeilenabstand im Literaturverzeichnis ist Einzeilig
% siehe Leitfaden S. 14
\AtBeginBibliography{\singlespacing}

%-----------------------------------
% Silbentrennung (Vgl. Leitfaden S.17) 
%-----------------------------------
\usepackage{hyphsubst}
\HyphSubstIfExists{ngerman-x-latest}{%
\HyphSubstLet{ngerman}{ngerman-x-latest}}{}
\usepackage[ngerman]{babel}
\usepackage{microtype}

\hyphenpenalty=5000 % Minimiert die Trennung
\exhyphenpenalty=5000 % Minimiert die Trennung am Zeilenende

%-----------------------------------
% Pfad fuer Abbildungen
%-----------------------------------
\graphicspath{{./}{./abbildungen/}}

%-----------------------------------
% Weitere Ebene einfügen
%-----------------------------------
\input{skripte/weitereEbene}

%-----------------------------------
% Paket für die Nutzung von Anhängen
%-----------------------------------
\usepackage{appendix}

%-----------------------------------
% Zeilenabstand 1,5-zeilig
%-----------------------------------
\usepackage{setspace}
\onehalfspacing

%-----------------------------------
% Absätze durch eine neue Zeile
%-----------------------------------
\setlength{\parindent}{0mm}
\setlength{\parskip}{0.8em plus 0.5em minus 0.3em}

\sloppy					%Abstände variieren
\pagestyle{headings}

%----------------------------------
% Präfix in das Abbildungs- und Tabellenverzeichnis aufnehmen, statt nur der Nummerierung (siehe Issue #206).
%----------------------------------
\KOMAoption{listof}{entryprefix} % Siehe KOMA-Script Doku v3.28 S.153
\BeforeStartingTOC[lof]{\renewcommand*\autodot{:}} % Für den Doppelpunkt hinter Präfix im Abbildungsverzeichnis
\BeforeStartingTOC[lot]{\renewcommand*\autodot{:}} % Für den Doppelpunkt hinter Präfix im Tabellenverzeichnis

%-----------------------------------
% Abkürzungsverzeichnis
%-----------------------------------
\usepackage[printonlyused]{acronym}

%-----------------------------------
% Symbolverzeichnis
%-----------------------------------
% Quelle: https://www.namsu.de/Extra/pakete/Listofsymbols.pdf
\usepackage[final]{listofsymbols}

%-----------------------------------
% Glossar
%-----------------------------------
\usepackage{glossaries}
\glstoctrue %Auskommentieren, damit das Glossar nicht im Inhaltsverzeichnis angezeigt wird.
\makenoidxglossaries
\input{abkuerzungen/glossar}

%-----------------------------------
% PDF Meta Daten setzen
%-----------------------------------
\usepackage[hyperfootnotes=false]{hyperref} %hyperfootnotes=false deaktiviert die Verlinkung der Fußnote. Ansonsten inkompaibel zum Paket "footmisc"
% Behebt die falsche Darstellung der Lesezeichen in PDF-Dateien, welche eine Übersetzung besitzen
% siehe Issue 149
\makeatletter
\pdfstringdefDisableCommands{\let\selectlanguage\@gobble}
\makeatother

\hypersetup{
    pdfinfo={
        Title={\myTitel},
        Subject={\myStudiengang},
        Author={\myAutor},
        Build=1.1
    }
}

%-----------------------------------
% PlantUML
%-----------------------------------
%\usepackage{plantuml}

%-----------------------------------
% Umlaute in Code korrekt darstellen
% siehe auch: https://en.wikibooks.org/wiki/LaTeX/Source_Code_Listings
%-----------------------------------
\lstset{literate=
	{á}{{\'a}}1 {é}{{\'e}}1 {í}{{\'i}}1 {ó}{{\'o}}1 {ú}{{\'u}}1
	{Á}{{\'A}}1 {É}{{\'E}}1 {Í}{{\'I}}1 {Ó}{{\'O}}1 {Ú}{{\'U}}1
	{à}{{\`a}}1 {è}{{\`e}}1 {ì}{{\`i}}1 {ò}{{\`o}}1 {ù}{{\`u}}1
	{À}{{\`A}}1 {È}{{\'E}}1 {Ì}{{\`I}}1 {Ò}{{\`O}}1 {Ù}{{\`U}}1
	{ä}{{\"a}}1 {ë}{{\"e}}1 {ï}{{\"i}}1 {ö}{{\"o}}1 {ü}{{\"u}}1
	{Ä}{{\"A}}1 {Ë}{{\"E}}1 {Ï}{{\"I}}1 {Ö}{{\"O}}1 {Ü}{{\"U}}1
	{â}{{\^a}}1 {ê}{{\^e}}1 {î}{{\^i}}1 {ô}{{\^o}}1 {û}{{\^u}}1
	{Â}{{\^A}}1 {Ê}{{\^E}}1 {Î}{{\^I}}1 {Ô}{{\^O}}1 {Û}{{\^U}}1
	{œ}{{\oe}}1 {Œ}{{\OE}}1 {æ}{{\ae}}1 {Æ}{{\AE}}1 {ß}{{\ss}}1
	{ű}{{\H{u}}}1 {Ű}{{\H{U}}}1 {ő}{{\H{o}}}1 {Ő}{{\H{O}}}1
	{ç}{{\c c}}1 {Ç}{{\c C}}1 {ø}{{\o}}1 {å}{{\r a}}1 {Å}{{\r A}}1
	{€}{{\EUR}}1 {£}{{\pounds}}1 {„}{{\glqq{}}}1
}

%-----------------------------------
% Kopfbereich / Header definieren
%-----------------------------------
\pagestyle{fancy}
\fancyhf{}
% Seitenzahl oben, mittig, mit Strichen beidseits
% \fancyhead[C]{-\ \thepage\ -}

% Seitenzahl oben, mittig, entsprechend Leitfaden ohne Striche beidseits
\fancyhead[C]{\thepage}
%\fancyhead[L]{\leftmark}							% kein Footer vorhanden
% Waagerechte Linie unterhalb des Kopfbereiches anzeigen. Laut Leitfaden ist
% diese Linie nicht erforderlich. Ihre Breite kann daher auf 0pt gesetzt werden.
\renewcommand{\headrulewidth}{0.4pt}
%\renewcommand{\headrulewidth}{0pt}

%-----------------------------------
% Damit die hochgestellten Zahlen auch auf die Fußnote verlinkt sind (siehe Issue 169)
%-----------------------------------
\hypersetup{colorlinks=true, breaklinks=true, linkcolor=darkblack, citecolor=darkblack, menucolor=darkblack, urlcolor=darkblack, linktoc=all, bookmarksnumbered=false, pdfpagemode=UseOutlines, pdftoolbar=true}
\urlstyle{same}%gleiche Schriftart für den Link wie für den Text

%-----------------------------------
% Start the document here:
%-----------------------------------
\begin{document}

\pagenumbering{Roman}								% Seitennumerierung auf römisch umstellen
\newcolumntype{C}{>{\centering\arraybackslash}X}	% Neuer Tabellen-Spalten-Typ:
%Zentriert und umbrechbar

%-----------------------------------
% Textcommands
%-----------------------------------
\input{skripte/textcommands}

%-----------------------------------
% Titlepage
%-----------------------------------
\input{kapitel/titelseite}

%-----------------------------------
% Vorwort (optional; bei Verwendung beide Zeilen entkommentieren und unter Inhaltsverzeichnis setcounter entsprechend anpassen)
%-----------------------------------
%\input{kapitel/vorwort/vorwort}
%\newpage

%-----------------------------------
% Inhaltsverzeichnis
%-----------------------------------
% Um das Tabellen- und Abbbildungsverzeichnis zu de/aktivieren ganz oben in Documentclass schauen
\setcounter{page}{2}
\addtocontents{toc}{\protect\enlargethispage{-20mm}}% Die Zeile sorgt dafür, dass das Inhaltsverzeichnisseite auf die zweite Seite gestreckt wird und somit schick aussieht. Das sollte eigentlich automatisch funktionieren. Wer rausfindet wie, kann das gern ändern.
\setcounter{tocdepth}{4}
\tableofcontents
\newpage

%-----------------------------------
% Abbildungsverzeichnis
%-----------------------------------
\listoffigures
\newpage
%-----------------------------------
% Tabellenverzeichnis
%-----------------------------------
\listoftables
\newpage
%-----------------------------------
% Abkürzungsverzeichnis
%-----------------------------------
% Falls das Abkürzungsverzeichnis nicht im Inhaltsverzeichnis angezeigt werden soll
% dann folgende Zeile auskommentieren.
\addcontentsline{toc}{section}{\abbreHeadingName}
\input{abkuerzungen/acronyms}
\newpage

%-----------------------------------
% Symbolverzeichnis
%-----------------------------------
% In Overleaf führt der Einsatz des Symbolverzeichnisses zu einem Fehler, der aber ignoriert werdne kann
% Falls das Symbolverzeichnis nicht im Inhaltsverzeichnis angezeigt werden soll
% dann folgende Zeile auskommentieren.
\phantomsection\addcontentsline{toc}{section}{\symheadingname}
\input{skripte/symbolDef}
\listofsymbols
\newpage

%-----------------------------------
% Glossar
%-----------------------------------
\printnoidxglossaries
\newpage

%-----------------------------------
% Sperrvermerk
%-----------------------------------
%\input{kapitel/anhang/sperrvermerk}

%-----------------------------------
% Seitennummerierung auf arabisch und ab 1 beginnend umstellen
%-----------------------------------
\pagenumbering{arabic}
\setcounter{page}{1}

%-----------------------------------
% Kapitel / Inhalte
%-----------------------------------
% Die Kapitel werden über folgende Datei eingebunden
\input{skripte/kapitelUebersicht.tex}

%-----------------------------------
% Apendix / Anhang
%-----------------------------------
\newpage
\section*{\AppendixName} %Überschrift "Anhang", ohne Nummerierung
\addcontentsline{toc}{section}{\AppendixName} %Den Anhang ohne Nummer zum Inhaltsverzeichnis hinzufügen

\begin{appendices}
% Nachfolgende Änderungen erfolgten aufgrund von Issue 163
\makeatletter
\renewcommand\@seccntformat[1]{\csname the#1\endcsname:\quad}
\makeatother
\addtocontents{toc}{\protect\setcounter{tocdepth}{0}} %
	\renewcommand{\thesection}{\AppendixName\ \arabic{section}}
	\renewcommand\thesubsection{\AppendixName\ \arabic{section}.\arabic{subsection}}
	\input{kapitel/anhang/anhang}
\end{appendices}
\addtocontents{toc}{\protect\setcounter{tocdepth}{2}}

%-----------------------------------
% Literaturverzeichnis
%-----------------------------------
\newpage

% Die folgende Zeile trägt ALLE Werke aus literatur.bib in das
% Literaturverzeichnis ein, egal ob sie zietiert wurden oder nicht.
% Der Befehl ist also nur zum Test der Skripte sinnvoll und muss bei echten
% Arbeiten entfernt werden.
%\nocite{*}

%\addcontentsline{toc}{section}{Literatur}

% Die folgenden beiden Befehle würden ab dem Literaturverzeichnis wieder eine
% römische Seitennummerierung nutzen.
% Das ist nach dem Leitfaden nicht zu tun. Dort steht nur dass 'sämtliche
% Verzeichnisse VOR dem Textteil' römisch zu nummerieren sind. (vgl. S. 3)
%\pagenumbering{Roman} %Zähler wieder römisch ausgeben
%\setcounter{page}{4}  %Zähler manuell hochsetzen

% Ausgabe des Literaturverzeichnisses

% Keine Trennung der Werke im Literaturverzeichnis nach ihrer Art
% (Online/nicht-Online)
%\begin{RaggedRight}
%\printbibliography
%\end{RaggedRight}

% Alternative Darstellung, die laut Leitfaden genutzt werden sollte.
% Dazu die Zeilen auskommentieren und folgenden code verwenden:

% Literaturverzeichnis getrennt nach Nicht-Online-Werken und Online-Werken
% (Internetquellen).
% Die Option nottype=online nimmt alles, was kein Online-Werk ist.
% Die Option heading=bibintoc sorgt dafür, dass das Literaturverzeichnis im
% Inhaltsverzeichnis steht.
% Es ist übrigens auch möglich mehrere type- bzw. nottype-Optionen anzugeben, um
% noch weitere Arten von Zusammenfassungen eines Literaturverzeichnisse zu
% erzeugen.
% Beispiel: [type=book,type=article]
\printbibliography[nottype=online,heading=bibintoc,title={\langde{Literaturverzeichnis}\langen{Bibliography}}]

% neue Seite für Internetquellen-Verzeichnis
\newpage

% Laut Leitfaden 2018, S. 14, Fussnote 44 stehen die Internetquellen NICHT im
% Inhaltsverzeichnis, sondern gehören zum Literaturverzeichnis.
% Die Option heading=bibintoc würde die Internetquelle als eigenen Eintrag im
% Inhaltsverzeicnis anzeigen.
%\printbibliography[type=online,heading=bibintoc,title={\headingNameInternetSources}]
\printbibliography[type=online,heading=subbibliography,title={\headingNameInternetSources}]

\newpage
\pagenumbering{gobble} % Keine Seitenzahlen mehr

%-----------------------------------
% Eigenständigkeitserklärung
%-----------------------------------
\section*{%
	\langde{Eigenständigkeitserklärung}
	\langen{Declaration of authorship}}
\langde{Hiermit versichere ich, dass ich die angemeldete Prüfungsleistung in allen Teilen eigenständig ohne Hilfe von Dritten anfertigen und keine anderen als die in der Prüfungsleistung angegebenen Quellen und zugelassenen Hilfsmittel verwenden werde. Sämtliche wörtlichen und sinngemäßen Übernahmen inklusive KI-generierter Inhalte werde ich kenntlich machen.

Diese Prüfungsleistung hat zum Zeitpunkt der Abgabe weder in gleicher noch in ähnlicher Form, auch nicht auszugsweise, bereits einer Prüfungsbehörde zur Prüfung vorgelegen; hiervon ausgenommen sind Prüfungsleistungen, für die in der Modulbeschreibung ausdrücklich andere Regelungen festgelegt sind.

Mir ist bekannt, dass die Zuwiderhandlung gegen den Inhalt dieser Erklärung einen Täuschungsversuch darstellt, der das Nichtbestehen der Prüfung zur Folge hat und daneben strafrechtlich gem. § 156 StGB verfolgt werden kann. Darüber hinaus ist mir bekannt, dass ich bei schwerwiegender Täuschung exmatrikuliert und mit einer Geldbuße bis zu 50.000 EUR nach der für mich gültigen Rahmenprüfungsordnung belegt werden kann.

Ich erkläre mich damit einverstanden, dass diese Prüfungsleistung zwecks Plagiatsprüfung auf die Server externer Anbieter hochgeladen werden darf. Die Plagiatsprüfung stellt keine Zurverfügungstellung für die Öffentlichkeit dar.}
\langen{I declare that this paper and the work presented in it are my own and has been generated by me as the result of my own original research without help of third parties. All sources and aids including AI-generated content are clearly cited and included in the list of references. No additional material other than that specified in the list has been used.

I confirm that no part of this work in this or any other version has been submitted for an examination, a degree or any other qualification at this University or any other institution, unless otherwise indicated by specific provisions in the module description.

I am aware that failure to comply with this declaration constitutes an attempt to deceive and will result in a failing grade. In serious cases, offenders may also face expulsion as well as a fine up to EUR 50.000 according to the framework examination regulations. Moreover, all attempts at deception may be prosecuted in accordance with § 156 of the German Criminal Code (StGB).

I consent to the upload of this paper to thirdparty servers for the purpose of plagiarism assessment. Plagiarism assessment does not entail any kind of public access to the submitted work.}


\par\medskip
\par\medskip

\vspace{5cm}

\begin{table}[H]
	\centering
	\begin{tabular*}{\textwidth}{c @{\extracolsep{\fill}} ccccc}
		\myOrt, \the\day.\the\month.\the\year
		&
		% Hinterlege deine eingescannte Unterschrift im Verzeichnis /abbildungen und nenne sie unterschrift.png
		% Bilder mit transparentem Hintergrund können teils zu Problemen führen
		\includegraphics[width=0.35\textwidth]{unterschrift}\vspace*{-0.35cm}
		\\
		\rule[0.5ex]{12em}{0.55pt} & \rule[0.5ex]{12em}{0.55pt} \\
		\langde{(Ort, Datum)}\langen{(Location, Date)} & \langde{(Eigenhändige Unterschrift)}\langen{(handwritten signature)}
		\\
	\end{tabular*} \\
\end{table}

\end{document}
